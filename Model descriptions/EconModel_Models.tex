\documentclass[fleqn]{article}
\usepackage{amsmath} 

\title{SKÖRDE OCH LAGRING AV SOCKERBETOR MODEL}
\author{William English, NBR Nordic Beet Research}
\date{September 2021}

\usepackage{Sweave}
\begin{document}
\Sconcordance{concordance:EconModel_Models.tex:EconModel_Models.Rnw:%
1 7 1 1 0 154 1}


\maketitle

\pagebreak

%%%%%%%%%%%%%%%%%%%%%%%%%%%%%%%%%%%%%%%%%%%
% I FÄLTET

\section{I FÄLTET}
  
  %%%%%%%%%%%%%%%%%%%%%%%%%%
  % SEN TILLVÄXT - POL
  \subsection{SEN TILLVÄXT - POL}

    \begin{equation}
      ST_P = 
      \begin{cases}
        0.010 & \text{if date < 15 Nov}\\
        0.005 & \text{if date $\geq$ 15 Nov and $\leq$ 30 Nov}\\
        0.000 & \text{if date > 30 Nov}
      \end{cases}
    \end{equation}
    
    Where:\\
    \hangindent=1.5cm
    ST = sen tillväxt (\%-enheter per dygnt)\\
    p = Pol (\%-enheter)

    \subsubsection{Källa}
    An educated guess

    \subsubsection{Planerade förbättringar}
    Build out a proper, weather depended growth model, that uses live data from the current year. This will probably follow the work done by the BBRO.
  
  %%%%%%%%%%%%%%%%%%%%%%%%%%
  % SEN TILLVÄXT - REN BETOR
  \subsection{SEN TILLVÄXT - REN BETOR}

    \begin{align}
      ST_{RB} = & 1.5735e^{-06} \times D^2_{10S} - 2.8177e^{-04} \times D_{10S} + 0.01244\\
      &\times {STV}
    \end{align}
    
    Where:\\
    \hangindent=1.5cm
    ST = sen tillväxt (\% per dygnt)\\
    RB = ren betor \\
    $D_{10S}$ = days after 10 September\\
    STV = Sen tillväxt potential, relativt till median (0.5, 1.25)

    \subsubsection{Planerade förbättringar}
    Build out a proper, weather depended growth model, that uses live data from the current year. This will probably follow the work done by the BBRO.

%%%%%%%%%%%%%%%%%%%%%%%%%%%%%%%%%%%%%%%%%%%
% UPPTAGNING

\pagebreak
\section{UPPTAGNING}

\subsection{SPILL}

  \begin{equation}
  \text{Spill} = 3.804e^{-4} \times RSB^2 + 4.508e^{-2} \times RSB + 0.25
  \end{equation}
  
  Where:\\
  \hangindent=1.5cm
  RSB = rotspetsbrot (\% > 2cm)\\
  
  \subsubsection{Källa}
  BBRO med multiplication faktor (x2) enligt erfarenhet inon industri.
  

%%%%%%%%%%%%%%%%%%%%%%%%%%%%%%%%%%%%%%%%%%%
% LAGRNING
\pagebreak
\section{LAGRING}

%%%%%%%%%%%%%%%%%%%%%%%%%%%%%%%%%%%%%%%%%%%
% LEVERANS

\pagebreak
\section{LEVERANS}

  \subsection{KOSTNADER}
  
  Kostand per ton orenheter (approxiamte)*
  
  \begin{equation}
    \frac{dSEK_{orenheter}}{dkm} =
    \begin{cases}
      0,841 & \text{km} < 145\\
      0,482 & \text{km $\geq 145$}
    \end{cases}
  \end{equation}

  Where:\\
  \hangindent=1.5cm
  Orenheter är ton\\
  km = kilometer\\
  Baskostnad (1km) = 23,74SEK/tn
  
  \subsubsection{Källa}
  *Data is taken from the 2020 price model. The above equations are only approximations. 
  Actual data is taken from the Nordic Sugar "Transportkostand för orenheter" table. 

%%%%%%%%%%%%%%%%%%%%%%%%%%%%%%%%%%%%%%%%%%%
% PRODUCTION AND PAYMENT

\pagebreak
\section{PRODUCTION OCH BETALNING}

  \subsection{RENHET}

  \begin{equation}
    \frac{dRenhet}{dD} =
    \begin{cases}
      0 & \text{D} < 20\\
      -0,0022*D + 0,0438 & \text{D $\geq 20$}
    \end{cases}
  \end{equation}

  Where:\\
  \hangindent=1.5cm
  Renhet är procent enheter\\
  D = day after harvest\\
  $R^2 = 0,9188$

    \subsubsection{Källa}
    Agrilog, Sweden, 2020. All varieties.
    
    \subsubsection{Planerade förbättringar}
    Link to variety. The model is currently biased towards varieties that probably lose a lot of cleanness late in a long-term storage campaign.

%%%%%%%%%%%%%%%%%%%%%%%%%%%%%%%%%%%%%%%%%%%
\end{document}
